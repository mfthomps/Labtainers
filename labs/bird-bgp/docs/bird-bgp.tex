\input{header}
\begin{document}

\begin{center}
{\LARGE Border Gateway Protocol}
\vspace{0.1in}\\
\end{center}


\section{Overview}
This exercise introduces Border Gateway Protocoal (BGP) fundamentals,
allowing students to configure BGP routers and view their behavior.
The lab uses Bird routers, which is an open source Linux-based router
implementation.

\subsection{Background}
This exercise assumes the student has received instruction on functions
of network routers, and BGP.  A tutorial on BGP can be found at:
\url{http://www.cs.fsu.edu/~xyuan/cis6930/APRICOT2004-BGP00.pdf}.  It is also assumed
that the student is familiar with basic Linux routing, e.g., as explored in the
routing-basics and routing-basics2 labs.

This lab exercise only touches on some of the most basic elements of BGP.

\section{Lab Environment}
This lab runs in the Labtainer framework,
available at http://my.nps.edu/web/c3o/labtainers.
That site includes links to a pre-built virtual machine
that has Labtainers installed, however Labtainers can
be run on any Linux host that supports Docker containers.

From your labtainer-student directory start the lab using:
\begin{verbatim}
    labtainer bird-bgp
\end{verbatim}
A link to this lab manual will be displayed, along with a link to the Bird router 
user guide.

\section{Lab topology}
The lab presents a simplified view of Internet routers implementing BGP.  Each router is connected to one or more notional
enterprises, respresented by a single computer.  In Figure \ref{fig:topology}, the routers are labled R1-R4.  Enterprises
are labeled E1-E5.  The routers exchange routing information and traffic over point-to-point ethernet links.
Each of these links has a network tap that forwards copies of traffic to the {\tt netmon} component (not pictured), 
which collects network traffic in files within its {\tt /taps} directory.

Note that E2 and E5 share router R2.  This might reflect that R2 belongs to an ISP, whose customers include
enterprise E2 and enterprise E5.  

\begin{figure}[H]
\begin{center}
\includegraphics [width=0.8\linewidth]{topo.png}
\end{center}
\caption{BGP Routing Topology}
\label{fig:topology}
\end{figure}

\section{Tasks}

\subsection{Explore}
The following items (among other), are available to explore the network:
\begin{itemize}
\item Wireshark and tcpdump are installed on the {\tt netmon} computer, use them to review the PCAP files found in
the {\tt /taps} directory.  When using Wireshark, if you encounter black or otherwise corrupt pulldown windows, try resizing the window,
or restarting the application.  The {\tt ctrl-r} key sequence will cause Wireshark to reload the PCAP file that is
currently being viewed, i.e., to see the latest traffic.
\item The {\tt traceroute} program is install on each enterprise computer.  Use that to observe the routes that
traffic may take between different enterprise computers.
\item Each router includes the Bird client, which you can start using {\tt sudo birdc}.  Use it to view routes and
protocol definitions.  Bird is configured via use of configuration files found at {\tt /usr/local/etc/bird.conf}. 
The bird service runs under systemd.  If you modify a configuration file, you may rstart bird using {\tt systemctl restart
bird}.
\end{itemize}

\subsection{Configure routers for R4}
By now you should have noticed that E4 cannot be reached from the other enterprise computers.  If not, go back and 
explore!

You are the network administrator for E4, and your task is to configure its router so that it will announce
the route to E4 to the other routers.  All the other routers are already configured to talk to R4, so you need
not modify their configurations.    Use the Bird user guide and the existing bird.conf files as examples 
to modify the R4 bird.conf file.  

To demonstration you have configured R4, you must successfully ping E1, E2 and E3 from E4.

Use Wireshark to observe the BGP routing information propagate to the other routers.

\section{Submission}
After finishing the lab, go to the terminal on your Linux system that was used to start the lab and type:
\begin{verbatim}
    stoplab 
\end{verbatim}
When you stop the lab, the system will display a path to the zipped lab results on your Linux system.  Provide that file to 
your instructor, e.g., via the Sakai site.

\copyrightnotice

\end{document}
