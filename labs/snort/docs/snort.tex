\input{header}
\begin{document}

\begin{center}
{\LARGE Routing Basics}
\vspace{0.1in}\\
\end{center}

\copyrightnotice

\section{Overview}
This exercise introduces the use of the snort system
to provide intrusion detection within a
Linux environment.  Students will configure simple 
snort rules and experiment with differences between
a network IDS and a host based IDS.  


\section{Lab Environment}
This lab runs in the Labtainer framework,
available at http://my.nps.edu/web/c3o/labtainers.
That site includes links to a pre-built virtual machine
that has Labtainers installed, however Labtainers can
be run on any Linux host that supports Docker containers.

From your labtainer-student directory start the lab using:
\begin{verbatim}
    start.py snort
\end{verbatim}
\noindent A link to this lab manual will be displayed.  

\section{Network Configuration}
This lab includes several networked computers as shown in Figure~\ref{fig:topology}.
When the lab starts, you will get virtual terminals, one connected to each
component.  The gateway is configured with {\tt iptables} to use NAT to translate
sources addresses of traffic from internal IP addresses, e.g., 192.168.1.1, to
our external address, i.e., 203.0.113.10.  The {\tt iptables} in the gateway also
routes web traffic (ports 80 and 443) to the web\_server component.

The gateway is also configured to mirror traffic that enters and leaves the gateway via the 203.0.113.10 link.  This
mirrored traffic is routed to the {\tt snort} component.  In other words, the snort component receives a copy
of each and every packet either entering or exiting via the 203.0.113.10 interface.

The snort component includes the Snort IDS utility.  It also includes Wireshark
to help you observe traffic being mirrored to the snort component.

The web\_server runs Apache and is configured to support SSL for web pages in the
{\tt www.example.com} domain.

The {\tt external\_ws} component includes the Firefox browser, and a local 
{\tt /etc/hosts} file that maps www.example.com to the external address of the
gateway, i.e., 203.0.113.10.  It also includes the nmap utility.


\begin{figure}[htb]
\begin{center}
\includegraphics [width=0.8\textwidth,natwidth=621,natheight=403]{snort.jpg}
\end{center}
\caption{Network topology for the snort lab}
\label{fig:topology}
\end{figure}

\section{Lab Tasks}
It is assumed that the student has received instruction or independent study on
the basic operation of Snort, and the general goals and mechanics of network intrusion detection.

Review the network topology.  In particular, consider the {\tt iptables} settings on the gateway.
These can be seen by reviewing the commands in {\tt /etc/rc.local}, which are used to
define the NAT translations and, critically for this lab, mirror traffic to the snort component. 

\subsection{Starting and stopping snort}
The Snort utility is installed on the snort component.  The home directory includes a {\tt start\_snort.sh}
script that will start the utility in \textit{Network Intrustion Dection Mode}, and display alerts
to the console.  Start snort:
\begin{verbatim}
   ./start_snort.sh
\end{verbatim}
When it comes time to stop snort, e.g., to add rules, simply use {\tt CTL-C}.

\subsection{Pre-configured Snort rules}
The Snort utility includes a set of pre-configured rules that create alerts for known
suspicious network activity. The configuration on the snort component is largely as it
exists after initial installation of the snort utility.  To see an example of 
some of the preconfigured rules, perform an nmap scan of www.example.com from the remote
workstation:
\begin{verbatim}
    sudo nmap www.example.com
\end{verbatim}
\noindent Note the alerts displayed at the snort console.  The rules that generate these alerts can be seen,
along with all rules, in {\tt /etc/snort/rules/}

\subsection{Write a simple (bad) rule}
Custom rules are typically added to the file at {\tt /etc/snort/rules/local.rules
Stop snort and add a rule that generates an alert for each tcp connection.  For example:
\begin{verbatim}
alert tcp any any -> any any (msg:"TCP detected"; sid:00002;)
\end{verbatim}

Then restart snort.  Test this rule by starting Firefox on the remote workstation:

\begin{verbatim}
    firefox www.example.com
\end{verbatim}

As you can see, the rule you wrote will overwhelm you with useless information.  So,
stop snort and delete the rule.

\subsection{Custom rule for CONFIDENTIAL traffic}
At the Firefox browser, which should be displaying the webpage from www.example.com,
we will display an unpublished web page that we know exists on the website.  In particular,
we have heard that the keen minds at the startup company have placed their confidential 
business plan at www.example.com/plan.html.  Take a look at it.

Now add a rule to your local.rules file on snort that will generate an alert 
whenever the text "CONFIDENTIAL" is sent out to the internet.  Be sure to give the
rule its own unique sid.  After adding the rule, restart snort.

On the browser at the remote workstation, clear your history (Menu / Preferences 
Security \& Privacy), and then refresh the plan.html page.  You should see an alert at
the snort console.

\subsection{Effects of encryption}
Back at the Firefox browser, again clear the browser history. Now alter the URL to make
use of the web server SSL function.  Change the url to https://www.example.com/plan.html.
Do you see a new snort alert?  Why?

One solution to this problem is to use a reverse proxy in front of the web server.  This
reverse proxy would handle the incoming web traffic and manage the SSL connections.
The web server would then receive only HTTP traffic, and outgoing traffic from the web server
could then be mirrored to the IDS.  We will not pursue that solution here.  

\subsection{Watching internal traffic}
Go to the ws2 (mary) component and run nmap:
\begin{verbatim}
    sudo nmap www.example.com
\end{verbatim}
What do you see on the snort component?  Why?

Go to the gateway component and edit the {\tt /etc/rc.local} script so that traffic to and
from mary's workstation is mirrored to the snort component.  You can do this by adding
these lines to the section of that file that defines the packet mirroring:
\begin{verbatim}
   iptables -t mangle -A PREROUTING -i $lan2 -j TEE --gateway 192.168.3.1
   iptables -t mangle -A POSTROUTING -o $lan2 -j TEE --gateway 192.168.3.1
\end{verbatim}

Then run the script to replace the {\tt iptables} rules with your new rules:
\begin{verbatim}
    sudo /etc/rc.local
\end{verbatim}
Now restart snort and again run nmap from mary's ws2 computer.


\subsection{Distinguishing traffic by address}
Start Firefox on mary's ws2 computer to view the confidential business plan:
\begin {verbatim}
    firefox www.example.com/plan.html
\end {vertabim}

The snort rules include two address fields, These addresses are used to check the
source from which the packet originated and the destination of the packet. The address
may be a single IP address or a network address. You can use \textit{any} keyword to apply a
rule on all addresses. The address is followed by a slash character and number of bits in
the  netmask.  For  example,  an  address  192.168.2.0/24  represents  C  class  network
192.168.2.0  with  24  bits  in  the  network  mask.


\section{Submission}
After finishing the lab, go to the terminal on your Linux system that was used to start the lab and type:
\begin{verbatim}
    stop.py snort
\end{verbatim}
When you stop the lab, the system will display a path to the zipped lab results on your Linux system.  Provide that file to 
your instructor, e.g., via the Sakai site.

\end{document}
