\input{header}
\begin{document}

\begin{center}
{\LARGE VPN Lab Exercise (host-to-gateway VPN)}
\vspace{0.1in}\\
\end{center}

\section{Overview}
This Labtainer exercise illustrates a simple
host-to-gateway vpn implemented with openvpn, and
a static shared key.

The example network, depicted in Figure \ref{fig:topology} is identical to that in
the "host-to-host VPN" (vpnlab) exercise, except
there is now a gateway in front of the server.
As with the previous exercise, the server
offers a simple HTTP service,  and the student
will use wget on the client to retrieve html
files from the server.


\section{Lab Environment}
This lab runs in the Labtainer framework,
available at http://nps.edu/web/c3o/labtainers.
That site includes links to a pre-built virtual machine
that has Labtainers installed, however Labtainers can
be run on any Linux host that supports Docker containers
or on Docker Desktop on PCs and Macs.

From your labtainer-student directory start the lab using:
\begin{verbatim}
    labtainer vpnlab2
\end{verbatim}
A link to this lab manual will be displayed.

The openvpn application is pre-installed
on the client and the gateway, and the
corresponding openvpn configuration files
already exist.  To create an encrypted tunnel,
the student only has to execute openvpn on
the client and the gateway.

\begin{figure}[H]
\begin{center}
\includegraphics [width=0.8\textwidth]{vpnlab2.png}
\end{center}
\caption{Network topology for vpnlab2}
\label{fig:topology}
\end{figure}

In this exercise, the student will observe
that the client is unable to reach the server
until the openvpn tunnel is established. And,
use of the gateway allows the client to name 
the server using the server's network address
rather than the network address associated with
the tunnel as was required in the host-to-host 
VPN lab exercise.

\section{Tasks}
\subsection{Attempt to retrieve data from the server}
Use tcpdump on the router to display network traffic:
\begin{verbatim}
    sudo tcpdump -n -XX -i eth0
\end{verbatim}

Use wget on the client to fetch the index.html file
\begin{verbatim}
    wget  http://<IPADDr>/index.html
\end{verbatim}
Where $<$IPADDR$>$ is the server network address, which you
can learn by running "ifconfig" on the server.

Observe that wget fails.  Use {\tt $<$ctrl$>$c} to exit wget.

\subsection{Start the VPN}
Start the openvpn program on the gateway:
\begin{verbatim}
    sudo openvpn --config gateway.conf --daemon
\end{verbatim}

Start the openvpn program on the client:
\begin{verbatim}
    sudo  openvpn --config client.conf --daemon
\end{verbatim}

Use wget again, just as was done previously.
Note the wget succeeds this time.
Note you are using the server's network address
rather than the address associated with an encrypted tunnel.
Observe the network traffic in tcpdump.

\section{Submission}
After finishing the lab, go to the terminal on your Linux system that was used to start the lab and type:
\begin{verbatim}
    stoplab 
\end{verbatim}
When you stop the lab, the system will display a path to the zipped lab results on your Linux system.  Provide that file to 
your instructor, e.g., via the Sakai site.

\copyrightnotice

\end{document}
