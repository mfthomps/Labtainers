

\paragraph{Configuring Apache Server.}
In the pre-built VM image, we use Apache server to host all the web
sites used in the lab. The name-based virtual hosting feature in
Apache could be used to host several web sites (or URLs) on the same
machine. A configuration file named {\tt default} in the directory
\url{"/etc/apache2/sites-available"} contains the necessary directives for the
configuration:

\begin{enumerate}
\item The directive {\tt "NameVirtualHost *"} instructs the web
  server to use all IP addresses in the machine (some machines
  may have multiple IP addresses).

\item Each web site has a {\tt VirtualHost} block that specifies the
  URL for the web site and directory in the file system that contains
  the sources for the web site. For example, to configure a web site
  with URL \url{http://www.example1.com} with sources in directory
  \url{/var/www/Example_1/}, and to configure a web site
  with URL \url{http://www.example2.com} with sources in directory
  \url{/var/www/Example_2/},
  we use the following blocks:

\begin{Verbatim}[frame=single]
<VirtualHost *>
    ServerName http://www.example1.com
    DocumentRoot /var/www/Example_1/
</VirtualHost>

<VirtualHost *>
    ServerName http://www.example2.com
    DocumentRoot /var/www/Example_2/
</VirtualHost>
\end{Verbatim}

\end{enumerate}

You may modify the web application by accessing the source in the
mentioned directories. For example, with the above configuration,
the web application \url{http://www.example1.com} can be changed by modifying
the sources in the directory \url{/var/www/Example_1/}.


