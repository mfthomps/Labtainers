\input{header}
\begin{document}

\begin{center}
{\LARGE Wireshark Introduction}
\vspace{0.1in}\\
\end{center}


\section{Overview}
This exercise introduces the the Wireshark network traffic analysis tool.
The student will use Wireshark to view network traffic captured in a ``PCAP''
file and locate a specific packet.  PCAP files contain
copies of network traffic stored in a format that can be processed
by various network analysis tools such as Wireshark and
tcpdump.  PCAP is short for ``packet capture''.

\subsection{Background}
This exercise assumes the student has received instruction TCP/IP networking.

This lab exercise only touches on some of the most basic features of Wireshark.
Details on using the tool can be found at \url{https://www.wireshark.org/docs/wsug\_html\_chunked/ChapterIntroduction.html}

\section{Lab Environment}
This lab runs in the Labtainer framework,
available at http://nps.edu/web/c3o/labtainers.
That site includes links to a pre-built virtual machine
that has Labtainers installed, however Labtainers can
be run on any Linux host that supports Docker containers.

From your labtainer-student directory start the lab using:
\begin{verbatim}
    labtainer wireshark
\end{verbatim}
A link to this lab manual will be displayed.

\section{Tasks}
\subsection{Explore}
Use the {\tt ls} command to view the content of the directory in the terminal that opened when you started the lab.
That {\tt telnet.pcapng} file contains the network traffic you will analyze.  Use
\begin{verbatim}
   file telnet.pcapng
\end{verbatim}
\noindent to view information about the file.  Note it is a \textit{pcap-ng} file, which is a variation of the PCAP file format
currently used as the default format for Wireshark and other analysis tools.

\subsection{Run wireshark to perform PCAP Analysis}

Start Wireshark using the {\tt wireshark} command.  Then use {\tt File->Open} to open the telnet.pcapng file.

Telnet is a communications protocol that allows a user to 
issue shell commands to a remote host. Telnet network 
traffic is not encrypted. Refer to the telnet lab for further background.

\subsection{Find a specific packet}

Locate the single packet which contains the password provided when the user attempted to use Telnet to login as the "john" user. 


\textbf{Hint}: If you type {\tt telnet.data} into the field that says ``Add a display filter'' (above the list of packets), the
tool will display only Telnet data packets.  Press {\tt return} to apply the filter.

Once you locate the single packet containing the invalid password, use {\tt File=>Export specified packets} to save the single
packet that you located.  Save the single packet as 'invalidpassword.pcapng'.  Be sure to select {\tt Selected packets only} in the Export dialog and be sure to get the file name exactly right.

After you save the packet, you might then use {\tt File=>Open} to open your new pcap file to confirm it contains the correct packet.

\section{Submission}
After finishing the lab, go to the terminal on your Linux system that was used to start the lab and type:
\begin{verbatim}
    stoplab 
\end{verbatim}
When you stop the lab, the system will display a path to the zipped lab results on your Linux system.  Provide that file to 
your instructor, e.g., via the Sakai site.

\copyrightnotice

\end{document}
