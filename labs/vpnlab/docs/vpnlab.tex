\input{header}
\begin{document}

\begin{center}
{\LARGE VPN Lab Exercise (host-to-host VPN)}
\vspace{0.1in}\\
\end{center}

\section{Overview}
This Labtainer exercise illustrates a simple
host-to-host vpn implemented with openvpn, and
a static shared key.

The example network includes a client and a 
server with a router between them.  The server
offers a simple HTTP service,  and the student
will use wget on the client to retrieve html
files from the server.

\section{Lab Environment}
This lab runs in the Labtainer framework,
available at http://nps.edu/web/c3o/labtainers.
That site includes links to a pre-built virtual machine
that has Labtainers installed, however Labtainers can
be run on any Linux host that supports Docker containers
or on Docker Desktop on PCs and Macs.

From your labtainer-student directory start the lab using:
\begin{verbatim}
    labtainer vpnlab
\end{verbatim}
A link to this lab manual will be displayed.

The openvpn application is pre-installed
on the client and the server, and the
corresponding openvpn configurtion files
already exist.  To create an encrypted tunnel,
the student only has to execute openvpn on
the client and the server.

The student will observe both unencrypted 
and encrypted network traffic using
tcpdump on the router.

\section{Tasks}

\subsection{Observe unencrypted traffic}
Use tcpdump on the router to display network traffic:
\begin{verbatim}
    sudo tcpdump -n -XX -i eth0
\end{verbatim}

Use wget on the client to fetch the index.html file
\begin{verbatim}
    wget  http://<IPADDr>/index.html
\end{verbatim}
Where $<$IPADDR$>$ is the server network address, which you
can learn by running "ifconfig" on the server.

Observe the network traffic from tcpdump.  Note plain-text
html in the data stream.

\subsection{Start the VPN}
Start the openvpn program on the server:
\begin{verbatim}
    sudo openvpn --config server.conf --daemon
\end{verbatim}

Start the openvpn program on the client:
\begin{verbatim}
    sudo  openvpn --config client.conf --daemon
\end{verbatim}

Use wget again, but this time using the server's
tunnel address, (which appears in interface "tun0"
of output from ifconfig on the server).

Observe the network traffic in tcpdump.

\section{Submission}
After finishing the lab, go to the terminal on your Linux system that was used to start the lab and type:
\begin{verbatim}
    stoplab 
\end{verbatim}
When you stop the lab, the system will display a path to the zipped lab results on your Linux system.  Provide that file to 
your instructor, e.g., via the Sakai site.

\copyrightnotice

\end{document}
